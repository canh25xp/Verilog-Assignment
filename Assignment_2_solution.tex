\documentclass{vhdl-assignment}
\usepackage{blindtext}

\title{Assignment 2 Solution}
\date{1 Oct 2023}

\begin{document}
\maketitle
\thispagestyle{fancy}

\begin{problem}{}
    $F(A,B,C,D)=\sum m(0,2,3,8,9,10,11,12,13,14,15)$
    
    \begin{table}[H]
        \centering
        \begin{displaymath}
            \begin{array}{c|c c c c|c}
                  & A & B & C & D & F \\
                \hline
                0  & 0 & 0 & 0 & 0 & 1 \\
                1  & 0 & 0 & 0 & 1 & 0 \\
                2  & 0 & 0 & 1 & 0 & 1 \\
                3  & 0 & 0 & 1 & 1 & 1 \\
                4  & 0 & 1 & 0 & 0 & 0 \\
                5  & 0 & 1 & 0 & 1 & 0 \\
                6  & 0 & 1 & 1 & 0 & 0 \\
                7  & 0 & 1 & 1 & 1 & 0 \\
                8  & 1 & 0 & 0 & 0 & 1 \\
                9  & 1 & 0 & 0 & 1 & 1 \\
                10 & 1 & 0 & 1 & 0 & 1 \\
                11 & 1 & 0 & 1 & 1 & 1 \\
                12 & 1 & 1 & 0 & 0 & 1 \\
                13 & 1 & 1 & 0 & 1 & 1 \\
                14 & 1 & 1 & 1 & 0 & 1 \\
                15 & 1 & 1 & 1 & 1 & 1 \\
            \end{array}
        \end{displaymath}
        \caption[short]{Truth Table}
    \end{table}
    
    \begin{figure}[H]
        \begin{subfigure}{0.5\textwidth}
            \centering
            \begin{karnaugh-map}[4][4][1][$D$][$C$][$B$][$A$]
                \manualterms{0,1,2,3,4,5,6,7,8,9,10,11,12,13,14,15}
            \end{karnaugh-map}
            \caption{Layout}
        \end{subfigure}
        \begin{subfigure}{0.4\textwidth}
            \centering
            \begin{karnaugh-map}[4][4][1][$D$][$C$][$B$][$A$]
                \minterms{0,2,3,8,9,10,11,12,13,14,15}
                \autoterms[0]
                \implicant{12}{10}
                \implicantcorner
                \implicantedge{3}{2}{11}{10}
            \end{karnaugh-map}
            \caption{Prime implicant}
        \end{subfigure}
        \caption{Karnaugh-map}
    \end{figure}
    
    $\Rightarrow F=A+B'D'+B'C$
    
    \begin{figure}[H]
        \centering
        \begin{circuitikz}
            \node (A) at (0,0) {$A$};
            \node (B) at (1,0) {$B$};
            \node (C) at (2,0) {$C$};
            \node (D) at (3,0) {$D$};
    
            \node[not port]                 (Not0)  at ($(D)    + (2, -2)$) {};
            \node[not port]                 (Not1)  at ($(D)    + (2, -4)$) {};
            \node[not port]                 (Not2)  at ($(D)    + (2, -6)$) {};
            \node[and port, anchor=in 1]    (And0)  at ($(Not0) + (2, 0)$) {};
            \node[and port]                 (And1)  at ($(And0) + (0, -3)$) {};
            \node[or port, number inputs=3] (0r0)   at ($(And0) + (3, -1)$) {};
    
            \draw (A) -- ($(A)+(0,-7)$); 
            \draw (B) -- ($(B)+(0,-7)$);
            \draw (C) -- ($(C)+(0,-7)$);
            \draw (D) -- ($(D)+(0,-7)$);
    
            \filldraw[black] ($(A)+(0,-1)$) circle (2pt) node[]{};
            \draw ($(A)+(0,-1)$) |- ($(A)+(9.5,-1)$) |- (0r0.in 1);
    
            \filldraw[black] ($(B)+(0,-2)$) circle (2pt) node[]{};
            \draw ($(B)+(0,-2)$) |- (Not0.in);
    
            \filldraw[black] ($(B)+(0,-4)$) circle (2pt) node[]{};
            \draw ($(B)+(0,-4)$) |- (Not1.in);
    
            \filldraw[black] ($(D)+(0,-6)$) circle (2pt) node[]{};
            \draw ($(D)+(0,-6)$) |- (Not2.in);
    
            \filldraw[black] ($(C)+(0,-3)$) circle (2pt) node[]{};
            \draw ($(C)+(0,-3)$) |- ($(C)+(4,-3)$) |- (And0.in 2);
    
            \draw (Not0.out) |- (And0.in 1);
            \draw (Not1.out) |- (And1.in 1);
            \draw (Not2.out) |- (And1.in 2);
            \draw (And0.out) |- (0r0.in 2);
            \draw (And1.out) |- (0r0.in 3);
    
            \node[right] (F) at (0r0.out) {$F$};
        \end{circuitikz}
        \caption{Circuit}
    \end{figure}

    \verilog[Verilog code for Problem 1]{src/Assignment_1_Problem_1.v}
\end{problem}

\newpage
\begin{problem}{Design a 4-to-1 Multiplexer}
    \begin{multicols}{2}
        \begin{table}[H]
            \begin{displaymath}
                \begin{array}{c c | c}
                    s0 & s1 & F \\
                    \hline
                    0  & 0  & i0 \\
                    0  & 1  & i1 \\
                    1  & 0  & i2 \\
                    1  & 1  & i3 \\
                \end{array}
            \end{displaymath}
            \centering
            \caption[short]{Truth Table for a 4-to-1 Multiplexer}
        \end{table}
        $\Rightarrow F=s0's1'i0 + s0s1'i1 + s0's1i2 + s0s1i3$
        
        \begin{figure}[H]
            \begin{circuitikz}
                \node[mux_4_to_1] (Mux0) {MUX};
                \node[right] (F) at (Mux0.rpin 1){$F$};
                \node[left] (i0) at (Mux0.lpin 1){$i0$};
                \node[left] (i1) at (Mux0.lpin 2){$i1$};
                \node[left] (i2) at (Mux0.lpin 3){$i2$};
                \node[left] (i3) at (Mux0.lpin 4){$i3$};
                \node[below] (s0) at (Mux0.bpin 1){$s0$};
                \node[below] (s1) at (Mux0.bpin 2){$s1$};
            \end{circuitikz}
            \centering
            \caption{4-to-1 Multiplexer}
        \end{figure}
    \end{multicols}
    
    \begin{figure}[H]
        \centering
        \begin{circuitikz}
            \node (i0) at (0,6) {$i0$};
            \node (i1) at (0,4) {$i1$};
            \node (i2) at (0,2) {$i2$};
            \node (i3) at (0,0) {$i3$};
    
            \node (s1) at (0,-4) {$s1$};
            \node (s0) at (0,-5) {$s0$};
    
            \node[and port, number inputs=3, anchor=in 1]   (And0)   at ($(i0) + (6, 0)$) {};
            \node[and port, number inputs=3, anchor=in 1]   (And1)   at ($(i1) + (6, 0)$) {};
            \node[and port, number inputs=3, anchor=in 1]   (And2)   at ($(i2) + (6, 0)$) {};
            \node[and port, number inputs=3, anchor=in 1]   (And3)   at ($(i3) + (6, 0)$) {};
    
            \node[not port, rotate=90] (Not1) at ($(s1) + (2,2)$) {};
            \node[not port, rotate=90] (Not0) at ($(s0) + (4,3)$) {};
    
            \node[or port, number inputs=4, anchor=in 1]    (Or0)    at ($(And0.out) + (2, 0)$) {};
    
            \draw (i0) -- (And0.in 1);
            \draw (i1) -- (And1.in 1);
            \draw (i2) -- (And2.in 1);
            \draw (i3) -- (And3.in 1);
    
            \draw (s1) -- ($(s1) + (2,0)$) -- (Not1.in);
            \draw (s0) -- ($(s0) + (4,0)$) -- (Not0.in);
    
            \draw (And0.out) -- (Or0.in 1);
            \draw (And1.out) |- (Or0.in 2);
            \draw (And2.out) |- ($(And2.out)+(0.5,0)$) |- (Or0.in 3);
            \draw (And3.out) |- ($(And3.out)+(1,0)$)   |- (Or0.in 4);
    
            \draw (Not0.out) |- (And0.in 3);
            \draw (Not0.out) |- (And2.in 3);
            \draw (Not1.out) |- (And0.in 2);
            \draw (Not1.out) |- (And1.in 2);
    
            \draw ($(s1)+(1,0)$) |- (And2.in 2);
            \draw ($(s0)+(3,0)$) |- (And1.in 3);
            \draw ($(And3.in 2)+(-5,0)$) |- (And3.in 2);
            \draw ($(And3.in 3)+(-3,0)$) |- (And3.in 3);
            
            \filldraw[black] ($(And1.in 2)+(-4,0)$) circle (2pt) node[]{};
            \filldraw[black] ($(And2.in 3)+(-2,0)$) circle (2pt) node[]{};
            \filldraw[black] ($(And3.in 2)+(-5,0)$) circle (2pt) node[]{};
            \filldraw[black] ($(And3.in 3)+(-3,0)$) circle (2pt) node[]{};
    
            \filldraw[black] ($(s1)+(1,0)$) circle (2pt) node[]{};
            \filldraw[black] ($(s0)+(3,0)$) circle (2pt) node[]{};
    
            \node[right] (F) at (Or0.out) {$F$};
        \end{circuitikz}
        \caption{4-to-1 Multiplexer Circuit}
    \end{figure}
    
    \verilog[Verilog code for 4-to-1 Multiplexer]{src/Assignment_1_Problem_2.v}
\end{problem}

\newpage
\begin{problem}{Adder Circuit}
    \subsection*{1.Half Adder}
    
    \begin{table}[H]
        \begin{displaymath}
            \begin{array}{c c | c c}
                X & Y & S & C\\
                \hline
                0  & 0  & 0 & 0 \\
                0  & 1  & 1 & 0 \\
                1  & 0  & 1 & 0 \\
                1  & 1  & 0 & 1 \\
            \end{array}
        \end{displaymath}
        \centering
        \caption[short]{Truth Table for Half Adder}
    \end{table}
    
    \begin{equation*}
        \Rightarrow
        \begin{cases}
            S = X'Y + XY' = X \oplus Y\\
            C = XY\\
        \end{cases}
    \end{equation*}
    
    \begin{figure}[H]
        \centering
        \begin{circuitikz}
            \node (X) at (0,2) {$X$};
            \node (Y) at (0,0) {$Y$};
    
            \node[xor port, anchor=in 1] (Xor) at ($(X)+(3,0)$) {};
            \node[and port, anchor=in 1] (And) at ($(Y)+(3,0)$) {};
    
            \draw (X) -- (Xor.in 1);
            \draw (Y) -- (And.in 1);
    
            \draw ($(X) + (1,0)$) |- (And.in 2);
            \draw ($(Y) + (2,0)$) |- (Xor.in 2);
    
            \filldraw[black] ($(X) + (1,0)$) circle (2pt);
            \filldraw[black] ($(Y) + (2,0)$) circle (2pt);
    
            \node[right] (S) at (Xor.out) {$S$};
            \node[right] (C) at (And.out) {$C$};
        \end{circuitikz}
        \caption{Half Adder Circuit}
    \end{figure}
    
    \verilog[Verilog code for Half Adder]{src/Assignment_1_Problem_3_Half_Adder.v}
    
    \newpage
    \subsection*{2.Full Adder}
    
    \begin{table}[H]
        \begin{displaymath}
            \begin{array}{c c c | c c}
                X & Y & Ci & S & Co\\
                \hline
                0  & 0  & 0 & 0 & 0 \\
                0  & 0  & 1 & 1 & 0 \\
                0  & 1  & 0 & 1 & 0 \\
                0  & 1  & 1 & 0 & 1 \\
                1  & 0  & 0 & 1 & 0 \\
                1  & 0  & 1 & 0 & 1 \\
                1  & 1  & 0 & 0 & 1 \\
                1  & 1  & 1 & 1 & 1 \\
            \end{array}
        \end{displaymath}
        \centering
        \caption[short]{Truth Table for Full Adder}
    \end{table}
    
    \begin{figure}[H]
        \begin{subfigure}{0.5\textwidth}
            \centering
            \begin{karnaugh-map}[4][2][1][$X$][$Y$][$Ci$]
                \minterms{1,2,4,7}
                \autoterms[0]
            \end{karnaugh-map}
            \caption{$S = Ci \oplus X \oplus Y $}
        \end{subfigure}
        \begin{subfigure}{0.4\textwidth}
            \centering
            \begin{karnaugh-map}[4][2][1][$X$][$Y$][$Ci$]
                \minterms{3,5,6,7}
                \autoterms[0]
                \implicant{7}{6}
                \implicant{3}{7}
                \implicant{5}{7}
            \end{karnaugh-map}
            \caption{$Co = XY + YCi + XCi$}
        \end{subfigure}
        \caption{Karnaugh-map Full Adder}
    \end{figure}
    
    \begin{figure}[H]
        \centering
        \begin{circuitikz}
            \node (X) at (0,2) {$X$};
            \node (Y) at (0,0) {$Y$};
            
            \node[xor port, anchor=in 1] (Xor0) at ($(X)+(2,0)$) {};
            \node[xor port, anchor=in 1] (Xor1) at ($(Xor0.out)+(2,0)$) {};
            \node[and port, anchor=in 1] (And0) at ($(Y)+(2,0)$) {};
            \node[and port, anchor=in 1] (And1) at ($(And0.out)+(2,0)$) {};
            \node[xor port, anchor=in 1] (Xor2) at ($(And1)+(3,0)$) {};
            
            \node (Ci) at ($(And1.in 2) + (-7,-1)$) {$Ci$};
    
            \draw (X) -- (Xor0.in 1);
            \draw (Y) -- (And0.in 1);
            \draw (Xor0.out) -- (Xor1.in 1);
    
            \draw ($(X) + (1,0)$) |- (And0.in 2);
            \draw ($(Y) + (2,0)$) |- (Xor0.in 2);
            \draw ($(Xor1.in 1) + (-0.5,0)$) |- (And1.in 1);
            \draw ($(And1.in 2) + (-1,0)$) -- (And1.in 2);
            \draw ($(And1.in 2) + (-1,0)$) |- (Xor1.in 2);
            \draw (Ci) -- ($(And1.in 2) + (-1,-1)$) |- ($(And1.in 2) + (-1,0)$);
            \draw (And1.out) -- (Xor2.in 1);
            \draw (And0.out) |- ($(And0.out) + (4,-2)$) |- (Xor2.in 2) ;
    
            \filldraw[black] ($(X) + (1,0)$) circle (2pt);
            \filldraw[black] ($(Y) + (2,0)$) circle (2pt);
            \filldraw[black] ($(Xor1.in 1) + (-0.5,0)$) circle (2pt);
            \filldraw[black] ($(And1.in 2) + (-1,0)$) circle (2pt);
    
            \node[right] (S) at (Xor1.out) {$S$};
            \node[right] (C) at (Xor2.out) {$Co$};
        \end{circuitikz}
        \caption{Full Adder Circuit}
    \end{figure}
    
    \verilog[Verilog code for Full Adder]{src/Assignment_1_Problem_3_Full_Adder.v}
    
    \newpage
    \subsection*{3.Ripple Carry 4-bit Adder}
    \verilog[Verilog code for Ripple Carry Adder]{src/Assignment_1_Problem_3_Ripple_Carry_Adder.v}
    
    \newpage
    \subsection*{4.Ripple Carry 8-bit Adder}
    \blindtext[3]
\end{problem}

\end{document}