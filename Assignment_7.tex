\documentclass{vhdl-assignment}

\title{Assignment 7}
\date{April 18, 2024}

\begin{document}
\maketitle
\thispagestyle{fancy}

\begin{problem}{Behavioral 4-to-1 Multiplexer}
    \begin{itemize}
        \item Design a 4-to-1 Multiplexer using Behavioral statements.
        \item Write testbench and simulate your design.
    \end{itemize}

    \begin{lstlisting}[language=Verilog, numbers=none]
module mux4_to_1 (out, i0, i1, i2, i3, s1, s0);
// write your design here
endmodule
\end{lstlisting}
    \note{You can re-use the testbench module that you created earlier for the Structural 4-to-1 Multiplexer in Assignment 3,
    or the Data Flow 4-to-1 Multiplexer in Assignment 6.
    The simulation result should be the same.}
\end{problem}

\begin{problem}{Behavioral 4-bit Counter}
    \begin{itemize}
        \item Design a 4-bit Counter using Behavioral statements.
        \item The counter contains an asynchronous, active-high reset signal.
        \item Write testbench and simulate your design.
    \end{itemize}
        \begin{lstlisting}[language=Verilog, numbers=none]
module counter(out, clock, reset);
// write your design here
endmodule
\end{lstlisting}
\end{problem}

\begin{problem}{Behavioral Flip-Flops}
    \begin{itemize}
        \item Design the following module :
        \begin{enumerate}
            \item A D flip-flop with an \emph{asynchronous}, active-low reset signal.
            \item A D flip-flop with an \emph{synchronous}, active-low reset signal.
            \item A SR flip-flop with an \emph{synchronous}, active-high reset signal.
            \item A T flip-flop with an \emph{synchronous}, active-low reset signal.
            \item A JK-flip-flop, with no reset signal.
        \end{enumerate}
        \item Write testbench and simulate your design.
    \end{itemize}
\end{problem}

\end{document}